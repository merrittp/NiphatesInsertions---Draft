\documentclass[../main.tex]{subfiles}
\begin{document}

\section{Discussion}
The primary finding from the sequencing of these two genomes is the novel presence of inverted repeats and insertions in the \emph{Niphates erecta} genome. These features are seen only in \emph{N. erecta} despite the close phylogenetic relationship to \emph{Niphates digitalis}. This indicates that the insertions are either newly gained in \emph{N. erecta} or newly lost in \emph{N. digitalis}. Sequencing additional \emph{Niphates} sponges might be useful in identifying the mode of acquisition.

The gene loss seen in both \emph{Niphates} species is novel in demosponges. Only one other species is known to have lost one of these genes, as \emph{Amphimedon queenslandica} is known to be missing \emph{atp9}. Loss of \emph{atp8} appears to be novel in demosponges.Gene loss is not uncommon in the history of mitochondrial genomes, and many mitochondrial genes can be found in the nuclear genome. Particularly, \emph{atp9} has been found in the nuclear genome of \emph{Amphimedon queenslandica}, and has a corresponding loss in the mitochondrial genome. This same trend is seen in \emph{Niphates digitalis}, as the transcriptome analysis shows transcripts of both \emph{atp8} and \emph{atp9}.

\end{document}