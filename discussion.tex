\documentclass[../main.tex]{subfiles}
\begin{document}

\section{Discussion}

The mitochondrial genomes for \emph{Niphates digitalis} and \emph{Niphates erecta} reveal many surprising characteristics, even in light of Porifera's documented diversity. However, when compared to other closely related sponges, gene loss, gene order, insertions, inverted repeats, and stem-loop elements make them unique among demosponges. These unique characteristics, when considered together, signify a turbulent - and recent - genetic upheaval. 

The gene loss seen in both \emph{Niphates} species is novel in demosponges. Normally, \emph{atp9} is found exclusively in the mitochondrial genomes of demosponges and glass sponges and no other animals. However, two other species of sponge is known to have lost \emph{atp9}, the demosponges \emph{Amphimedon queenslandica} and \emph{Neopetrosia proxima}. Interestingly, \emph{A. queenslandica} is the closest related species to both \emph{Niphates digitalis} and \emph{Niphates erecta}, and shares with them accelerated genome evolution. More interestingly, all four species that share loss of \emph{atp9} have also lost multiple mt-tRNA genes. However, this trend is not seen across all sponge species missing mitochondrial tRNAs.

Generally, gene loss is not uncommon in the history of mitochondrial genomes, and many mitochondrial genes can be found in the nuclear genome. Keeping with this trend, \emph{atp9} has been found in the nuclear genome of \emph{Amphimedon queenslandica}. Due to flanking inverted terminal repeats on either side of \emph{A. queenslandica}'s nuclear \emph{atp9} gene, transposon-mediated transport has been proposed as the mode of gene movement. In the case of \emph{N. digitalis} and \emph{N. erecta}, the nuclear genomes have not been published, and therefore there is no evidence of transposon-mediated transport of \emph{atp9} in either species. That being said, the presence of stem-loop elements and inverted repeats in the mitochondrial genome of \emph{N. digitalis} lends support to the hypothesis of a similar mode of movement in that species. 

The evidence together indicates the possibility of an early, lineage-specific loss of \emph{atp9} in an ancestor of \emph{Amphimedon queenslandica} and both \emph{Niphates} species, and then a second, universal loss of \emph{atp9} in the branches leading to animals. 

Concurrent with the loss of \emph{atp9} in both \emph{Niphates} species is the loss of \emph{atp8}. It appears that \emph{atp8} is more frequently lost than other mitochondrial genes, as it has been independently lost in as some molluscs, arrow worms, nematodes, some vertebrates, and all known species of placozoans. In the case of phylum Porifera, the loss of \emph{atp8} is not novel, as certain glass sponges have also lost it, but the loss of \emph{atp8} in \emph{Niphates digitalis} and \emph{Niphates erecta} is the first reported loss of \emph{atp8} in any demosponge. 

In \emph{Niphates digitalis}, the transcriptome results indicate that \emph{atp8} shares a similar fate to \emph{atp9}. The discovery of \emph{atp8} transcripts with no corresponding mitochondrial sequence suggest that \emph{atp8} was also moved to the nuclear genome, and therefore does not constitute a true genetic loss. 

This gene loss, when considered with \emph{Niphates}'s one-of-a-kind gene order, signifies a turbulent - and relatively recent - genetic upheaval. The gene clusters mentioned in the previous section are seen in various combinations across most species of haplosclerid sponges, and rarely are the clusters disrupted. Most variation comes from cluster order arrangements, and tRNA gene position variation. The \emph{Niphates} species show extensive rearrangement, with clusters being flipped, relocated, or inserted into other clusters. As none of this rearrangement is seen in \emph{A. queenslandica}, it is mostly the result of recent development in the genetic history of these \emph{Niphates} species. Whether this arrangement is seen in other \emph{Niphates} species, of which there are several, has not yet been determined. Further sequencing projects would be needed to understand how widespread this gene arrangement is. 



\end{document}