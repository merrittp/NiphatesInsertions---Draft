\documentclass[../main.tex]{subfiles}
\begin{document}

\newpage
\section{Introduction}
Lacking both true tissues and bilateral symmetry, the study of phylum Porifera offers a unique perspective on basal Metazoans and the subsequent evolution of key traits. Much emphasis has therefore been placed on exploring the genetic diversity of these species. Though sequencing nuclear genomes of sponges is difficult, a multitude of novel mitochondrial genomes have been published for this group.

The genomes have shown remarkable variation in terms of their gene structure, gene content, size, and organization. Genes are commonly lost or gained, but one of the more novel traits discovered is the loss of mt-tRNAs in many sponge species, particularly those in Class Demosponge. However, the loss of mt-tRNAs is sporadic and seems to be species-specific. For all the sponge species with sequenced mitochondrial genomes, no pattern of loss has been identified, and the impact the loss has on cellular processes and genome evolution has not been widely studied. Determining how sponges cope with their mitochondrial tRNA loss is also of medical interest, seeing as many human diseases are attributed to mutational loss of mt-tRNAs. In order to effectively study mt-tRNA loss and it's impact in Porifera, more sponge species missing mt-tRNAs need to be identified and sequenced.

To this end, we report the publication of the mitochondrial genomes of two closely related demosponge species, \emph{Niphates digitalis} and \emph{Niphates erecta}, both of which are missing all but 4 mt-tRNAs. Despite their close phylogenetic relationship, the two mt-genomes show striking differences in genome architecture. While such diversity has been documented between species of different genus and with a larger evolutionary distance, the genomes \emph{Niphates digitalis} and \emph{Niphates erecta} showcase an interesting example of intense genomic change on a short evolutionary time-scale. 

\end{document}