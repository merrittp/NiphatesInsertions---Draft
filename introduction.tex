\documentclass[../main.tex]{subfiles}
\begin{document}

\newpage
\section{Introduction}

In 2007, Erpenbeck et al published the mitochondrial genome of the demosponge \emph{Amphimedon queenslandica}. It differed drastically from the few other mitochondrial sponge genomes published, and introduced the concept that mitochondrial genome diversity was far greater in Porifera than anticipated. In 2009, Srivastava et al published \emph{A. queenslandica}'s nuclear genome, the first nuclear genome of a sponge to be sequenced and published. Being the only sponge with published nuclear and mitochondrial genomes, \emph{A. queenlandica}'s place as the model organism for sponge studies was solidified. Since then, research into \emph{A. queenslandica} has expanded, and no other sponge nuclear genome has been sequenced to the depth and quality of \emph{A. queenlandica}. Many other sponge mt-genomes have since been sequenced and published, but \emph{A. queenlandica} remains the reference to which other sponge mt-genomes are compared. Even so, \emph{Amphimedon queenslandica} appears to be the outlier when sponge mt-genomes are considered together.

It was previously thought that all mitochondrial genomes were fairly uniform. They were short and compact, with little non-coding DNA. They contained a fixed set of protein-coding genes, two ribosomal RNA genes, and a complete set of 22 tRNA genes to assist in translation. This led to the myth of a "standard mitochondrial genome." This uniformity, however, has been shown to not be universal. Although the majority of bilateral mt-genomes keep with this consistency, several non-bilateral lineages deviate, including Porifera. Even so, the mt-genome of \emph{A. queenslandica} stands out compared to other sequenced mt-genomes of sponges. It has a long non-coding region with repeat elements, that might constitute a putative control region, a trait not found in other sponges. It also displays many stem-loop elements both within and outside protein-coding genes, which has only been seen sporadically across Porifera. Most notable in the \emph{Amphimedon queenslandica} mt-genome is the loss of the gene \emph{atp9}. Evidence in the nuclear genome indicates that \emph{atp9} moved to the nuclear genome via transposition. The same cannot be said for \emph{A. queenslandica}'s seven missing mt-tRNA genes, which seem to have disappeared entire from both genomes.

The loss of mitochondrial genes, like that seen in \emph{A. queenslandica}, is seldom found in bilaterian animals such as mammals, and is objectively more common in non-bilaterians. More often, however, non-bilaterian genomes have additional components. One example is the repetitive invasion of hairpin loops seen in many sponge mitochondrial genomes and \emph{A. queenslandica}. This invasion has happened independently multiple times in sponges, and almost always appear to be recent in the organism's evolutionary history. The origin of these invasions are unknown and the respective function of the stem-loops has not been studied, though some hypothesize the stem-loops might function as regulatory mechanisms, particularly when clustered together. The result is almost always a substantial increase in the size of mitochondrial genomes. In addition, \emph{atp9}, a gene that encodes for a subunit of the mitochondrial F0-ATP synthase, is found only in sponges, and has not been found in any other lineage outside of plants and the phylum Porifera. Other examples of novel genes in non-bilaterians are \emph{tatC} in Oscarellidae, \emph{mutS} in Octocorallia, and \emph{polB} in Medusoza. 

Though mt-gene loss is highly uncommon in non-bilaterians, when it does happen, it most commonly happens in the form of missing mt-tRNA genes. As such, \emph{A. queenslandica}'s loss of \emph{atp9}, a protein-coding gene, is unique. In sponges, the novel and lineage specific loss of mt-tRNA genes constitutes the majority of cases where the mt-genome has experienced gene loss. For all the sponge species with sequenced mitochondrial genomes, no pattern of mt-tRNA gene loss has been identified, and the impact the loss has on cellular processes and genome evolution has not been widely studied. Previous research has shown mitochondrial tRNA synthetases can be lost in parallel with mitochondrial tRNAs genes. However, this is not always the case, as it does appear that certain mt-tRNA genes and mt-tRNA synthetases are retained for unknown reasons. The factors that promote this loss of tRNA genes and the mechanisms by which it occurs are also unknown. This loss is of scientific interest, as it was originally believed that mt-tRNA genes were critical for normal cellular function. In particular, numerous human diseases are caused by mutations in human mt-tRNA genes. The discovery that some animals are able to lose some, or all, of their mitochondrial tRNA genes make them a prime target for study.

It is important to note that of all the sponge species with published mt-genomes, the majority do not have missing tRNAs, missing genes, or independent invasions of stem-loop elements. \emph{Amphimedon queenslandica} has all the aforementioned characteristics, and to differing degrees, so do the other members of Clade B, of which \emph{A. queenlandica} is a member. The fact that \emph{Amphimedon queenslandica} is considered the model sponge organism, yet displays these rare and novel elements, makes its evolutionary history and collection of traits a necessary point of study.

To investigate the evolutionary history of these traits in \emph{A. queenslandica}, we analyzed the four available mitochondrial genomes of Clade B sponges - \emph{Xestospongia testudinaria}, \emph{Xestospongia muta}, \emph{Neopetrosia proxima}, and \emph{Amphimedon queenslandica} itself. In addition, we report the newly sequenced mitochondrial genomes of two other Clade B species, \emph{Niphates digitalis} and \emph{Niphates erecta}. Both species are the most closely related to \emph{Amphimedon queenslandica} and have lost \emph{atp9}, all but four mt-tRNA genes, and an additional mt-gene. In addition, the mt-genomes of \emph{N. digitalis} and \emph{N. erecta} show drastic differences, despite their highly similar gene content, structure, and organization, as well as their close phylogenetic relationship. These mt-genomes, when compared with \emph{Amphimedon queenslandica}, showcase the genome evolution across Clade B that culminated in the extreme divergence seen in \emph{Amphimedon queenslandica} and other members of Clade B.

\end{document}