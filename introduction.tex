\documentclass[../main.tex]{subfiles}
\begin{document}

\newpage
\section{Introduction}
The mitochondrial genomes of mammals are frequently characterized as small, with a well-conserved set of genes, and a common circular structure. However,  mitochondrial genetic diversity in Metazoa has been shown to be greater than represented by mammals. The mitochondrial genomes of bilaterial organisms still maintain their strong conservation, but the mitochondrial genomes from non-bilateral phyla Cnidaria, Ctenophora, Placoza, and Porifera seem to hold the majority of mitochondrial genetic diversity. 

Of these, phylum Porifera offers a unique perspective on basal Metazoans and the subsequent evolution of key traits such as symmetry and tissue development. Their ancient evolutionary history, combined with their wide geographically range and availability make them useful for evaluating diversity. The most well-studied and largest group in phylum Porifera is Class Demospongiae - referred to simply as demosponges from here on out - which represents over 80\% of all sponge species on earth. 

Their genomes have shown remarkable variation in terms of their gene structure, gene content, size, and organization. Genes are commonly lost or gained, but one of the more novel traits discovered is the loss of mt-tRNAs in many sponge species, particularly those in Class Demosponge. 

However, the loss of mt-tRNAs is sporadic and seems to be species-specific. For all the sponge species with sequenced mitochondrial genomes, no pattern of loss has been identified, and the impact the loss has on cellular processes and genome evolution has not been widely studied. Previous research has shown corresponding changes in the mitochondria and mitochondrial genome when mitochondrial tRNAs are lost; in particular mitochondrial tRNA synthetases can be lost in parallel with mitochondrial tRNAs genes. The factors that promote this loss of tRNA genes and the mechanisms by which it occurs are also unknown. In order to effectively study mt-tRNA loss and it's impact in Porifera, more sponge species missing mt-tRNAs need to be identified and sequenced.

To this end, we report the publication of the mitochondrial genomes of two closely related species, \emph{Niphates digitalis} and \emph{Niphates erecta}. These two species are part of order Haplosclerida, of class Demospongiae, and share a close phylogenetic relationship with each other.

Despite their close phylogenetic relationship, the two mt-genomes show striking differences in genome architecture. While such diversity has been documented between species of different genus and with a larger evolutionary distance, the genomes \emph{Niphates digitalis} and \emph{Niphates erecta} showcase an interesting example of intense genomic change on a short evolutionary time-scale. 

\end{document}