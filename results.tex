\documentclass[../main.tex]{subfiles}
\begin{document}

\section{Results}

\subsection{Genome description}
Two new genomes generated in the genus Niphates - Niphates erecta, and Niphates digitalis

N. erecta - rnl, cox2, apt6, cox3, cox1, nad1, cob, rns, nad5, nad2, nad4, nad6, nad3

N. digitalis - rnl, cox2, atp6, cox3, cox1, nad1, cob, rns, nad5, nad2, nad4, nad6, nad3

Same gene order, same tRNAs. Slightly differing lengths. N. erecta has many more intergenic regions. Both species missing atp 8 and atp 9, seen in other species. Still retain atp6. 

\emph{N. digitalis} displays a much more compacted mt-genome structure as compared to \emph{N. erecta}, with few intergenic regions. A large intergenic region falls between \emph{nad2} and \emph{nad4}.

Insert genome description and pictures here

\subsection{Insertions, deletions, inverted repeats}
When compared to N. digitalis, N. erecta had a multitude of insertions present in it's genome. During analysis, 80 insertions found, though some might be unamplified regions. The size of insertions ranged in size from a few base pairs to large sections up to 650 bps long, and the majority fell in intergenic regions. That being said, certain protein-coding genes also showed novel insertions - \emph{cox2, nad1, nad5}. In addition, \emph{rnl} and \emph{rns}

\subsection{tRNA content and sythetases}
Missing all but 4 mitochondrial tRNAs

Viraj is looking into synthetases to see how many are still present.

Pett paper talks about tRNA synthetase loss in species missing tRNAs. 

\subsection{Accelerated sequence evolution}
Both species have accelerated rates of evolution. Interestingly, N. digitalis has a slightly higher rate of evolution as compared to N. erecta, even though N. erecta has those additional insertsion.

\end{document}