\documentclass[../main.tex]{subfiles}
\begin{document}

\section{Results}

\subsection{Genome description}
The two new genomes generated for this study were found to be strikingly similar. Both are circular-mapping molecules, and contain 11 protein-coding gene, two rRNA genes, and four tRNA genes. Both species are missing \emph{atp8} and \emph{atp9}, as is common in other demosponges. The gene order for both species is the same, as seen in Figure 1, but the two genomes show a length differences. The majority of this length difference can be attributed to the increased number of intergenic regions seen in \emph{N. erecta}.  A large intergenic region falls between \emph{nad2} and \emph{nad4} in \emph{N. digitalis}. The \emph{N. erecta} and \emph{N. digitalis} genomes are 25525bps and 18285bps, respectively.

The two genomes also show the nucleotide composition. The \emph{N. erecta} genome has a GC and AT content of 0.3640 and 0.6365, while \emph{N. digitalis} has a GC and AT content of 0.3640 and 0.6360. 

Overall, \emph{N. digitalis} displays a much more compacted mt-genome structure as compared to \emph{N. erecta}, with fewer intergenic regions.


\subsection{Insertions, deletions, inverted repeats}
The primary difference between the two genomes lies in the novel insertions identified in the \emph{N. erecta} mt-genome. When compared to \emph{N. digitalis, N. erecta} had a multitude of insertions present in it's genome. Eighty insertions have been identified, though some might be unamplified regions. The size of insertions ranged in size from a few base pairs to large sections up to 650 bps long, and the majority fell in intergenic regions. That being said, certain protein-coding genes also showed novel insertions - \emph{cox2, nad1, nad5}. In addition, \emph{rnl} and \emph{rns}

\subsection{tRNA content and sythetases}
Missing all but 4 mitochondrial tRNAs

Viraj is looking into synthetases to see how many are still present.

Pett paper talks about tRNA synthetase loss in species missing tRNAs. 

\subsection{Accelerated sequence evolution}
Both species have accelerated rates of evolution. Interestingly, N. digitalis has a slightly higher rate of evolution as compared to N. erecta, even though N. erecta has those additional insertsion.

\end{document}