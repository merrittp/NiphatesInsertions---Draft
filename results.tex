\documentclass[../main.tex]{subfiles}
\begin{document}

\section{Results}

The sequencing of these two genomes revealed they are highly similar to each other in many aspects, and both assemble into circular-mapping molecules, containing 11 protein-coding genes, two rRNA genes, and four tRNA genes, see Figure 1 for detailed diagram.

\subsection{Gene Loss}
Interestingly, both \emph{Niphates digitalis} and \emph{Niphates erecta} are missing the genes for \emph{atp8} and \emph{atp9}, which code for subunits of mitochondrial ATP synthetase. Loss of \emph{apt9} has only been seen in three other demosponge, the most well-studied of which is \emph{Amphimedon queenslandica}, while \emph{atp8} has been lost multiple times in animals, but has not been conclusively reported as lost in any other sponge species. 

Functional transcripts of \emph{atp8} and \emph{atp9} were found in the transcriptome of \emph{Niphates digitalis}, suggesting a move to the nuclear genome. An analysis of the \emph{atp9} nucleotide sequences from \emph{Niphates digitalis} and \emph{Amphimedon queenslandica} as compared to \emph{atp9} sequences in other species showed... An analysis of the \emph{atp8} sequence from \emph{Niphates digitalis}'s transcriptome showed...

\subsection{Gene Order}
A key finding was discovered when the assembled gene order of the new genomes were evaluated. While the gene order for both species was the same, (see Figure X), the gene order was strikingly different than other demosponge mitochondrial genomes previously reported. 

To better understand the differences, gene orders from 62 mitochondrial genomes from G3 and G4 demosponges were compared to the gene orders of \emph{Niphates digitalis} and \emph{Niphates erecta}. Generally, the mitochondrial genes formed clusters of genes in a specific order that were almost always found together. The order of the clusters change, but little variation to the cluster identities were seen in the other demosponge gene orders.

The first gene cluster is \emph {rns - rnl}. The second gene cluster is \emph{cox2 - tRNAs - atp8 - atp6 - tRNAs - cox3 - tRNAs - cob - atp9}. The third gene cluster is \emph{cox1 - tRNAs - nad1}, the fourth gene cluster is \emph{nad4 - nad6 - nad3}, and the fifth gene cluster is \emph{nad2 - nad5}. The gene \emph{nad4l} moves positions more frequently than others, but is most commonly found before \emph{cox1}. tRNA genes positions maintain a less strict position, though some patterns are generally seen. The genes for G and V fall between \emph{rns} and \emph{rnl}, with F in front of \emph{rns}, between \emph{cox3} and \emph{cob} are the genes for Q, W, N, and L1, and between \emph{cox1 and nad1} are the genes for S2, D, and C.

The five clusters alternate in pattern, with no cluster of NADH genes situated next to each other. As seen in Figure X, G3 sponges show a pattern of the \emph{rnl} group, \emph{cox1} group, \emph{nad2} group, the \emph{cox2} group, and the \emph{nad4} group. Individual genes are typically separated by tRNA genes, but the identity of tRNA genes between genes is not consistent. 

The mitochondrial genomes for both \emph{Niphates} species do not match either pattern identified in G3 or G4, despite being most closely related to the G3 sponge \emph{Amphimedon queenslandica}. Multiple clusters have been rearranged, and the cluster orders seen in G3 and G4 are not seen in \emph{Niphates}. The gene order for both species is seen in Figure 2b. To begin, \emph{rnl} and \emph {rns} are not found together, and instead are found with 8 and 6 genes on either side of them. The gene \emph{nad4l} has been separated from the \emph{cox1} group and is instead found by the \emph{cox2} group. The \emph{cox2} group appears to have undergone serious rearrangement, as evidenced by the \emph{cox1} group appearing between \emph{cox3} and \emph{cob}. In addition, the \emph{nad2} group has been flipped, with \emph{nad5} appearing before \emph{nad2}. Finally, breaking the pattern of the NADH gene groups not appearing side-by-side, the \emph{nad4} group is found next to the flipped \emph{nad2} group. 

\subsection{Insertions}
The primary difference between the two genomes lies in the novel insertions identified in the \emph{N. digitalis} mt-genome. When compared to \emph{N. erecta, N. digitalis} has a multitude of insertions present. Eighty insertions have been identified. Insertion sequences were analyzed using NCBI BLAST and found to have no close sequence relation, regardless of parameters. This discounts these insertions as possible contamination. The size of insertions ranged in size from a few base pairs to large sections up to 650 bps long, and have a similar nucleotide composition to the genome as whole. Additionally, the majority of insertions fall into intergenic regions, as shown in Figure 1. Though insertions were not exclusively found in intergenic regions, as genes also showed novel insertions - \emph{cox2, atp6, nad1, nad5, rnl}, and \emph{rns}, of which \emph{cox2, atp6, nad1,} and \emph{nad5} are protein-coding.

As a few of these insertions fall into protein-coding genes, it was important to understand the impact these insertions are having on down-stream products. To begin, the insertions were categorized on if they are in-frame or out-of-frame. There are four insertions found in \emph{cox2} and all are in-frame insertions, and the overall nucleotide addition to \emph{cox2} is in-frame. The gene \emph{atp6} has 5 insertions, one of which is out-of-frame. The overall nucleotide addition to \emph{atp6} is also out-of-frame. Three of the genes involved in NADH production, \emph{nad1, nad2,} and \emph{nad5} also have insertions - \emph{nad1} has two insertions, both of which are in-frame, \emph{nad2} has one insertion, which is in-frame, and \emph{nad5} has four insertions. Three of the four insertions are out-of-frame, and the overall nucleotide addition is out-of-frame for \emph{nad5}. 

As two of these genes - \emph{atp6} and \emph{nad5} - have out-of-frame insertions, the next step was to determine if these insertions are present in downstream products. **INSERT TRANSCRIPTOME ANALYSIS HERE WHEN COMPLETED**

Regardless of the impact these insertions and deletions have on proteins and downstream functions, they do account for the discrepancy in genomes size between species. The \emph{N. digitalis} and \emph{N. erecta} genomes are 25525 bps and 18285 bps, respectively. The majority of this length difference can be attributed to the increased number of intergenic regions and insertions seen in \emph{N. digitalis}. Intergenic regions account for 24.2\% of \emph{N. digitalis}'s genome, as opposed to 13.77\% in \emph{N. erecta}, and insertions in protein-coding areas account for an additional [INSERT PERCENTAGE HERE] of \emph{N. digitalis}'s mitochondrial genome.

However, the two genomes show an identical nucleotide composition.  The \emph{N. digitalis} genome has a GC and AT content of 36.40\% and 63.65\%, while \emph{N. erecta} has a GC and AT content of 36.40\% and 63.60\%. Without the insertions being considered, \emph{N. digitalis} has a GC and AT content of XX and YY respectively. This indicates that the insertions are not impacting the nucleotide composition, as one might anticipate if the insertions came from another organism. 

\subsection{Inverted Repeats}

Further analysis of \emph{N. digitalis}'s insertions and intergenic regions found that many contain inverted repeat sequences. The locations of these can be seen in Figure Y. Inverted repeats have previously been identified in other sponge species, and their function is unknown. These inverted repeats were isolated from the genome, and like the insertions, analyzed with NCBI BLAST. They returned no close sequence relations under any parameters. Alignments of the \emph{Niphates} inverted repeats showed that majority of repeats have the same general sequence, see Figure Z, and when folded form hairpin-like structures.

The inverted repeats can be categorized into four different motifs, the structures of which can be seen in Figure Z. The majority of inverted repeats contain motif 1. Motif 1 contains a highly conserved 11-nucleotide stem with a 4 nucleotide loop. Before and after the stem-loop is not conserved. Motif 2 shows a similar pattern, though it is less common. It contains a well-conserved 12 nucleotide stem with a 3 nucleotide loop at the top. In addition, bases before the stem-loop are typically conserved. Motif  is much smaller, with a 5 nucleotide stem and a 6 nucleotide loop. 

\subsection{tRNA content and sythetases}
One of the novel traits seen in many sponge mitochondrial genomes is a lack of mitochondrial tRNAs. \emph{Niphates digitalis} and \emph{Niphates erecta} are no exceptions. Both genomes had the same four mitochondrial tRNAs - Y, I, M, and W. These tRNAs fall into the same place in the genome, with Y, I, and M between \emph{nad4l} and \emph{cox2}, and W being between \emph{cob} and \emph{rns}.

Alignment of these tRNA genes with mt-tRNA genes from other demosponge species shows these tRNAs to be closely related to the same tRNA gene in other species, with no abnormal structures. 

Previous research has shown that 

Viraj is looking into synthetases to see how many are still present.

\subsection{Accelerated sequence evolution}
When protein-coding sequences were aligned with other demosponge species, both species showed an accelerated rate of evolution. A long branch connected \emph{N. digitalis} and \emph{N. erecta} to others of their group, and the analysis gave a short evolutionary distance between the two species. Interestingly, \emph{N. digitalis} has a slightly higher rate of evolution as compared to \emph{N. erecta}, even though \emph{N. erecta} has additional insertions.

\end{document}